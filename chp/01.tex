
\chapter{Introdução}
\label{cap:intro}

Este documento mostra como usar o \LaTeX\ com a classe \textsf{imeufg} para formatar teses, dissertações, monografias e relatórios de conclusão de curso, segundo o padrão adotado pelo Instituto de Informática da UFG. Este documento e a classe \textsf{imeufg} foram, em grande parte, copiados e adaptados da classe \textsf{thesisPUC} e do texto de Thomas Lewiner \cite{Lew2002} que descreve a sua utilização.

 \LaTeX\ é um sistema de editoração eletrônica muito usado para produzir documentos científicos de alta qualidade tipográfica. O sistema também é útil para produzir todos os tipos de outros documentos, desde simples cartas até livros completos.

Se você precisar de algum material de apoio referente ao \LaTeX, dê uma olhada em um dos sites do Comprehensive TEX Archive Network (CTAN). O site está em \href{http://www.ctan.org/}{www.ctan.org}. Todos os pacotes podem ser obtidos via \textsf{FTP} \href{ftp://www.ctan.org/}{ftp://www.ctan.org} e existem vários servidores em todo o mundo. Eles podem ser encontrados, por exemplo, em \href{ftp://ctan.tug.org/}{ftp://ctan.tug.org} (EUA), \href{ftp://ftp.dante.de/}{ftp://ftp.dante.de} (Alemanha), \href{ftp://ftp.tex.ac.uk/}{ftp://ftp.tex.ac.uk} (Reino Unido).

Você pode encontrar uma grande quantidade de informações e dicas na página dos usuários brasileiros de \LaTeX\ (\TeX-BR). O endereço é \href{http://biquinho.furg.br/tex-br/}{http://biquinho.furg.br/tex-br/}.
Tanto no CTAN quanto no \TeX-BR estão disponíveis bons documentos em português sobre o \LaTeX. Em particular no CTAN, está disponível uma introdução bastante completa em português: \href{http://www.ctan.org/tex-archive/info/lshort/portuguese-BR/lshortBR.pdf}{CTAN:/tex-archive/info/lshort/portuguese-BR/}. No \TeX-BR também existe um documento com exemplos de uso de \LaTeX\ e de vários pacotes: \href{http://biquinho.furg.br/tex-br/doc/LaTeX-demo/}{http://biquinho.furg.br/tex-br/doc/LaTeX-demo/} . O objetivo é ser, através de exemplos, um guia para o usuário de \LaTeX\ iniciante e intermediário, podendo, ainda, servir como um guia de referência rápida para usuários avançados.

Se você quer usar o \LaTeX\ em seu computador, verifique em quais sistemas ele está disponível em \href{http://www.ctan.org/tex-archive/systems/}{CTAN:/tex-archive/systems}. Em particular para \textsf{MS Windows}, o sistema gratuito \href{http://www.miktex.org/}{MikTeX}, disponível no CTAN e no site \href{http://www.miktex.org/}{www.miktex.org} é completo e atualizado de todas as opções  que você poderia precisar para editar o seu texto.

O estilo \textsf{imeufg} se integra completamente ao \LaTeXe. Uma tese, dissertação ou monografia escrita no estilo padrão do \LaTeX\ para teses (estilo \verb|report|) pode ser formatada em 15 minutos para se adaptar às normas da UFG.

O estilo \textsf{imeufg} foi desenhado para minimizar a quantidade de texto e de comandos necessários para escrever a sua dissertação. Só é preciso inserir algumas macros no início do seu arquivo \LaTeX, precisando os dados bibliográficos da sua dissertação (por exemplo o seu nome, o titulo da dissertação\ldots). Em seguida, cada página dos elementos pré-textuais será formatada usando macros ou ambientes específicos. O corpo do texto é editado normalmente. Finalmente, as referências bibliográficas podem ser entradas manualmente (via o comando \verb|\bibitem| do \LaTeX\ padrão) ou usando o sistema BiBTeX (muito mais recomendável). Neste caso, os arquivos \verb|imeufg.bst| e \verb|abnt-alf.bst| permitem a formatação das referências bibliográficas segundo as normas da UFG.

