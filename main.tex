\documentclass[dissertacao]{imeufg}
% Opções da classe imeufg (ao usar mais de uma, separe por vírgulas)
%   [tese]         -> Tese de doutorado.
%   [dissertacao]  -> Dissertação de mestrado (padrão).
%   [monografia]   -> Monografia de especialização.
%   [relatorio]    -> Relatírio final de graduação.
%   [abnt]         -> Usa o estilo "abnt-alf" de citação bibliográfica.
%   [nocolorlinks] -> Os links de navegação no texto ficam na cor preta.
%                     Use esta opção para gerar o arquivo para impressão
%                     da versão final do seu texto!!!
%--------------------------------------------------------------------------%
%[INÍCIO DO DOCUMENTO]-----------------------------------------------------%
%--------------------------------------------------------------------------%
\begin{document}
%--------------------------------------------------------------------------%
%[AUTOR, TÍTULO E DATA DE DEFESA]------------------------------------------%
%--------------------------------------------------------------------------%
\autor{\textless Nome do Autor do Trabalho\textgreater}
\autorR{\textless Nome Reverso do Autor do Trabalho\textgreater}
\titulo{\textless Título do Trabalho\textgreater}
\subtitulo{\textless Subtítulo do Trabalho\textgreater}
\cidade{Goiânia}
\dia{\textless Dia\textgreater} % \dia{03}
\mes{\textless Mês\textgreater} % \mes{10}
\ano{\textless Ano\textgreater} % \ano{1985}
%--------------------------------------------------------------------------%
%[ORIENTADOR]--------------------------------------------------------------%
%--------------------------------------------------------------------------%
\orientador{\textless Nome do Orientador\textgreater}
\orientadorR{\textless Nome Reverso do Orientador\textgreater}
%--------------------------------------------------------------------------%
%[ORIENTADORA]-------------------------------------------------------------%
%--------------------------------------------------------------------------%
\orientadora{\textless Nome da Orientadora\textgreater}
\orientadoraR{\textless Nome Reverso da Orientadora\textgreater}
%--------------------------------------------------------------------------%
%[COORIENTADOR]------------------------------------------------------------%
%--------------------------------------------------------------------------%
\coorientador{\textless Nome do Co-orientador\textgreater}
\coorientadorR{\textless Nome Reverso do Co-orientador\textgreater}
%--------------------------------------------------------------------------%
%[COORIENTADORA]-----------------------------------------------------------%
%--------------------------------------------------------------------------%
\coorientadora{\textless Nome da Co-orientadora\textgreater}
\coorientadoraR{\textless Nome Reverso da Co-orientadora\textgreater}
%--------------------------------------------------------------------------%
%[INSTITUIÇÃO E PROGRAMA]--------------------------------------------------%
%--------------------------------------------------------------------------%
\universidade{Universidade Federal de Goiás}
\uni{UFG}
\unidade{Instituto de Matemática e Estatística}
\departamento{\textless Nome do Departamento\textgreater} % Unidades com mais de um depto.
\universidadeco{\textless Nome da Universidade do Co-orientador\textgreater}
\unico{\textless Sigla da Universidade do Co-orientador\textgreater}
\unidadeco{\textless Nome da Unidade acadêmica do Co-orientador\textgreater}
\programa{Matemática}
\concentracao{\textless Área de Concentração\textgreater}
%--------------------------------------------------------------------------%
%[ELEMENTOS PRÉ-TEXTUAIS]--------------------------------------------------%
%--------------------------------------------------------------------------%
\capa    % Gera o modelo da capa externa do trabalho
\publica % Gera a autorização para publicação em formato eletrônico
\rosto   % Primeira folha interna do trabalho
% \input{./pre/pre_aprovacao}
\input{pre/pre_direitos}
\input{pre/pre_dedicatoria}
\input{pre/pre_agradecimentos}
\input{pre/pre_epigrafe}
\input{pre/pre_resumo}
\input{pre/pre_abstract}
%--------------------------------------------------------------------------%
%--------------------------------------------------------------------------%
%--------------------------------------------------------------------------%
\tabelas[figtabalgcod]
%Opções:
%nada [] -> Gera apenas o sumário
%fig     -> Gera o sumário e a lista de figuras
%tab     -> Sumário e lista de tabelas
%alg     -> Sumário e lista de algoritmos
%cod     -> Sumário e lista de cídigos de programas
%
% Pode-se usar qualquer combinação dessas opções.
% Por exemplo:
%  figtab       -> Sumário e listas de figuras e tabelas
%  figtabcod    -> Sumário e listas de figuras, tabelas e
%                  cídigos de programas
%  figtabalg    -> Sumário e listas de figuras, tabelas e algoritmos
%  figtabalgcod -> Sumário e listas de figuras, tabelas, algoritmos e
%                  códigos de programas
%--------------------------------------------------------------------------%
%[CAPÍTULOS]---------------------------------------------------------------%
%--------------------------------------------------------------------------%
\input{chp/01}
\input{chp/02}
\input{chp/03}
%--------------------------------------------------------------------------%
%[BIBLIOGRAFIA]------------------------------------------------------------%
%--------------------------------------------------------------------------%
\cleardoublepage
\arial
\bibliography{ref/main}
%--------------------------------------------------------------------------%
%[APÊNDICES]---------------------------------------------------------------%
%--------------------------------------------------------------------------%
\apendices
\input{pos/01}
\input{pos/02}
\end{document}
